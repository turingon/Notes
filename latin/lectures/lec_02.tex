\lecture{2}{2025 February 27 16:01}{Nouns \& Their Cases}
In this topics we will explore the topic of nouns now in latin
threre are 5 groups of nouns and they are called declensions\\\\ 
Remember we said latin is an inflected language that means we looked
at the endings of word to vital clues about their purpose in a sentence \\\\ 
so declensions are merely groups based on how these nouns end \\ \\
\begin{tabular}{lll}
  & Declensions & \\
  1st & stella,stellae & star \\ 
  2nd & hortus,horti & garden \\ 
  3rd & rex,regis & king \\ 
  4th & manus,manus & hand \\ 
  5th & res,rei & thing \\
\end{tabular}\\ \\
\begin{center}
  \huge Cases: \\
\end{center}
\large{Nominative:}
The subject of your word for example \\ 
Caeser had a dog named Brutus \\ \\ 
\large{Genitive:}
Used for possessions no not demonic possessions Think of \\ 
the dog of Caeser is named Brutus \\ 
Caesar's dog is named Brutus \\ 
Canis Caesaris Brutum nominatur \\ \\
\large{Dative:} The indirect object of a sentence \\ 
Caesar gives the ball to the dog \\ 
Caesar pilam cani dat \\ \\ 
\large{Accusative:} The direct object receives the object of a verb for example \\ 
Caesar gives the ball to the dog \\ 
Caesar pilam cani dat \\ \\ 
\large{Ablative:} Versatile used with prepositions like by under and with ab,sub and cum
Also have ablative of means or instruments for example \\ 
Cicero writes my means of a stylus \\ 
Cicero scribit stylo \\ \\ 
\large{Vocative:} Used to express the noun of direct address; that is the person \\ 
Musa, mihi causas memora \\ 
O Muse, the causes and crimes relate \\ 
O tempora! O mores! \\
Oh the time we live in! Oh, the corruption \\ \\ 
\large{Locative:} Used when speaking about the place, cities, and small islands, but rare. We tend to called
it the home Rome rule Looks like Genitive: \\ 
domi(at home) \\ 
Romae(at Rome) \\
