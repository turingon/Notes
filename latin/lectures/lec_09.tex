\lecture{9}{2025 March 01 14:25}{Latin Verbs}
Probably one of the most complicated topics to covers is the 
Latin Verb \\ \\
Because there is so many endings, and these depends on:
\begin{itemize}
  \item Person (Who is doing the action)
  \item Number (How many of us are there)
  \item Tense  (When are we doing it)
  \item Voice  (Active or passive) and
  \item Mood   (Indicative or subjunctive (wish that i would))
\end{itemize}
Well latin is an inflected language. I can't say that enough 
it likes to wrap every word up in a neat little package \\ 
\begin{center}
  The Latin VERB is \underline{no} exception
\end{center} 
Take the verb amo, amare, amavi, amatum ... meaning to love 
\begin{center}
  \begin{tabular}{lll}
    & Singular & Plural \\ 
    1st & am -o & am -a -mus \\ 
    2nd & am -a -s & am -a -tis \\ 
    3rd & am -a -t & am -a -nt \\
  \end{tabular} \\
\end{center} 
Note that the theme vowel is "a" \\ \\ 
Some say that Latin is a backwards language. \\ 
It's true, so if you backwards you will find it more efficient \\ 
Here, let me show you what I mean \\ \\ 
Let us choose the word amat $\longrightarrow$ am -a -t
\begin{enumerate}[I]
  \item We have the \underline{suffix} -t, that tell you that the person 
    and number, here is third person singular 
  \item Next we have the \underline{infix}, it tells you the the tense here
    we don't have one but if you were to travel forwards or backwards in time like the doctor
    you might do \textbf{ba} for the Imperfect and \textbf{bo} for the Future
  \item Next you have the "\textbf{theme vowel}". Here it is -\textbf{a}- which indicates the mood, Indicative, ie. things that
    \underline{do happen} versus the Subjunctive things that \underline{might} happen or you 
    would wish to happen. 
  \item Finally you have the stem "am" 
\end{enumerate}
Take the verb moneo, monere, monui, monitum ... meaning to warn
\begin{center}
  \begin{tabular}{lll}
    & Singular & Plural \\ 
    1st & mon -e -o  & mon -e -mus \\ 
    2nd & mon -e -s & mon -e -tis \\ 
    3rd & mon -e -t & mon -e -nt \\ 
  \end{tabular}
\end{center}
Now I promised to you let's look at the principal parts of the Verbs
\begin{center}
  \begin{tabular}{llll}
    1 & 2 & 3 & 4 \\ 
    amo & amare & amavi & amatus \\ 
  \end{tabular}
\end{center}

