\lecture{16}{2025 March 2 10:39}{Adjactives}
Where would we be without \underline{Adjactives} They make
things more interesting. They provide the color and chracter
for the subject and objects of out sentences \\ \\ 
They are a funny sort. They like to run around behind the 
nouns they describe, dressing up like them. Think of them 
as the groupies of Latin language. \\ \\ 
So how do Latin adjactives work? Well like the Latin nouns they
modify, they decline. Most take either the 1st or 2nd Declension
endings (magnus -a -um "great"), or the 3rd Declension 
(felix,felicis "lucky") \\ \\ 
All adjactives \textbf{must} agree with the nouns they describe 
in \textbf{gender}, \textbf{number}, and \textbf{case}
\begin{center}
  \Large Adjactive Placemet
\end{center}
Where do they go? Good question. Most follow behind the nouns
they modify. For example: \\ 
He is a \textbf{good} horse $\Longrightarrow$ Equus est \textbf{bonus}. \\ \\ 
Only a few adjactives are \underline{so great} they appear before
their nouns For example: \\ 
A \textbf{great} work $\Longrightarrow$ \textbf{Magnus} optus \\ \\ 
The ones that may give you trouble are:
\begin{enumerate}[I]
  \item 3rd Declension Adjactives + 1st and 2nd Declension nouns or
  \item 3rd Declension Nouns + 1st and 2nd Declension Adjactives 
\end{enumerate}
Here let me show you what I mean \\
The good man $\Longrightarrow$ homo \textbf{bonus} \\ \\ 
Some adjactives get a little too \textbf{big} for their britches
becoming those rock star nouns they used to idealize we call 
them substantives \\ \\ 
What if we wanted to say a sharp sword \\ 
we will use nominative case of sword which is gladius + nominative 
case of sharp which is acer then our sentence will be \textbf{gladius acer} \\ \\ 
So how can we say by sharp sword in this time we use ablative case of each word \\ 
then our sentence will be \textbf{gladio acri}

