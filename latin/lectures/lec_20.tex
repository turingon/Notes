\lecture{20}{2025 March 9 10:03}{Pluperfect Tense}
\begin{center}
  {\textbf{\huge Past Tense}}
\end{center}
In latin, we have seen the \textbf{Imperfect Tense},
which is used to indicate an action that has occured 
\textbf{sometime} in the past but we are \textbf{not quite 
sure when } or \textbf{if it is complete} for example \\ \\
\begin{center}
   \begin{tabular}{llll}
    ama\textbf{bam} & I was loving & ama\textbf{bamus} & we were loving \\ 
    ama\textbf{bas} & You were loving & ama\textbf{batis} & you all were loving \\ 
    ama\textbf{bat} & he/she/it was loving & ama\textbf{bant} & they were loving 
   \end{tabular} 
\end{center}
\begin{center}
  \begin{tabular}{llll}
    amav\textbf{i} & I loved & amav\textbf{imus} & We loved \\ 
    amav\textbf{isti} & You loved & amav\textbf{istis} & You loved \\ 
    amav\textbf{it} & he/she/it loved & amav\textbf{erunt} & They loved 
  \end{tabular}
\end{center}
\begin{center}
  {\textbf{\huge Pluperfect Tense}}
\end{center}
The \textbf{Pluperfect tense} extends even further back in time
than either Imperfect or Perfect \\ 
\begin{center}
  {\textbf{\huge Pluperfect Form}}
\end{center}
The Pluperfect = Perfect stem + the Imperfect form of verb to be ...
\begin{center}
  \textbf{\Large amo,amare,\underline{amavi},amatum}
\end{center}
\begin{center}
  \begin{tabular}{llll}
    amav\textbf{eram} & I had loved & amav\textbf{eramus} & we had loved \\ 
    amav\textbf{eras} & you had loved & ama\textbf{eratis} & you had loved \\ 
    amav\textbf{erat} & he/she/it loved & ama\textbf{erant} & they had loved
  \end{tabular}
\end{center}
