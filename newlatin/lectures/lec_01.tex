\lecture{1}{2025 April 07 16:02}{Day 1}
Today objectives are
\begin{enumerate}
  \item Learn latin pronunciation
  \item Start 1st declension noun endings
  \item Learn present tense of 1st conjugation verbs
  \item get familiar with latin sentence structure
\end{enumerate}
\section*{Latin Pronunciation}
Latin uses consisten phonics like:\\ 
V = pronounced like english W \\ 
e.g., "Veni" = "Weni"\\\\
C = always hard like K \\ 
e.g., "Caesar" = "Kaiser"\\\\
AE = pronounced like "eye"\\ 
e.g., "aetas" = "eye-tahs"\\\\
and there is a short types of vowels which they have a 
line above the letter we pronounce them same with the vowels
but with a short sound \\\\
\large{\textbf{Example}}\\
Vita brevis, ars longa. == "Life is short, art is long"
\section*{1st Declension}
In latin there are 5 declension and today we will learn 1st 
declension today the 1st declension mostly consist of femine
nouns but there are some rare masculine nouns for example poēta meaning poet
the pattern is highly regular also this making it very easy \\
\begin{tabular}{lll}
  Case & Singular & Plural \\
  Nominative & -a & -ae \\ 
  Genitive & -ae &  -ārum \\ 
  Dative & -ae & -īs \\ 
  Accusative & -am & -ās \\ 
  Ablative & -ā & -īs \\ 
  Vocative & -a & -ae
\end{tabular}\\ \\ \\
\large{\textbf{Examples}}\\
puella,puellae (girl,girls)
\begin{itemize}
  \item Nominative Singular: puella (the girl)
  \item Nominative Plural: puellae (the girls)
  \item Genitive Singular: puellae (of the girl)
  \item Genitive Plural: puellārum (of the girls)
\end{itemize}
dea,deaea (goddess,goddesses)
\begin{itemize}
  \item Nominative Singular: dea (the goddess)
  \item Genitive Singular: deae (of the goddess)
\end{itemize}
insula,insulae (island,islands)
\begin{itemize}
  \item Nominative Singular: insula (the island)
  \item Accusative Plural: inslās (the islands)
\end{itemize}
\subsection*{Steps for Declining a 1st Declension Noun:}
\paragraph{Find the stem:}
to get the stem of the word you should look at the Genitive 
singular form for example in puella the genitive is puellae 
and the stem is puell-
\paragraph{Add endings:}
Add endings to stem based on appropiate case and number 
\subsection*{Important Notes}
\begin{itemize}
  \item The Vocative case is often same as the nominative 
  for the 1st Declension 
  \item Gender in the 1st declension usually is feminine but
  you have to aware some masculine type words like poēta,agricola
  and nauta
\end{itemize}
\subsection*{Practice Sentences}
\begin{itemize}
  \item Puella amat librum. === The girl loves the book
  \item Puellae in horto ambulant === The girls are walking
  in the garden 
  \item Deae amicae sunt === The goddesses are friends
\end{itemize}
\subsection*{Practice Exercise}
Decline these 1st declension nouns:
\begin{enumerate}
  \item fīlia,fīliae (daughter) 
  \item villa,villae (house)
  \item stella,stellae (star)
\end{enumerate}
\section*{1st Conjugation Present Tense}
1st conjugation verbs in latin are verbs that end in -āre in 
the infinitive form e.g. amare = "to love" \\ 
the stem of these verb is the infinitive form minus the -re ending  \\ \\
\large{\textbf{Example}} \\ 
Infinitive:amare \\ 
Stem:am
\subsection*{1st Conjugation Verbs in the Present Tense}
In latin we use present tense for actions happening now or 
actions that are generally true (like habits or repeated actions) \newline
The endings for 1st Conjugation verbs in the present tense are: \\ \\
\begin{tabular}{lll}
  Person & Singular & Plural \\
  1st Person & -ō (I) & -āmus (we) \\ 
  2nd Person & -ās (you) & -ātis (you all) \\ 
  3rd Person & -at (he/she/it) & -ant (they) \\
\end{tabular}
\subsection*{Steps to Conjugate a 1st Conjugation Verb in Present Tense}
\begin{enumerate}
  \item Find the stem 
  \item Add the correct present tense ending
\end{enumerate}
\subsection*{Example Verbs in Present Tense(1st Conjugation)}
1. amare (to love)

    1st person singular: amō (I love)

    2nd person singular: amās (you love)

    3rd person singular: amat (he/she/it loves)

    1st person plural: amāmus (we love)

    2nd person plural: amātis (you all love)

    3rd person plural: amant (they love) \\ \\
2. laborare (to work)

    1st person singular: laborō (I work)

    2nd person singular: laborās (you work)

    3rd person singular: laborat (he/she/it works)

    1st person plural: laborāmus (we work)

    2nd person plural: laborātis (you all work)

    3rd person plural: laborant (they work) \\\\
3. spectare (to look at, to watch)

    1st person singular: spectō (I watch)

    2nd person singular: spectās (you watch)

    3rd person singular: spectat (he/she/it watches)

    1st person plural: spectāmus (we watch)

    2nd person plural: spectātis (you all watch)

    3rd person plural: spectant (they watch)
\subsection*{Practice Exercise:}
Conjugate these 1st conjugation verbs in the present tense: \\
dare (to give) \\ 
cantare (to sing) \\ 
vocare (to call) \\
