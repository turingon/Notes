\lecture{03}{2025 April 09 15:35}{Day 3}
\section*{What is 2nd Declension}
The 2nd declension is a group of nouns charactized by their 
genitive singular ending in -ī (e.g., servus,servī) it 
includes mostly masculine and neuter nouns
\subsection*{Masculine Nouns}
common nominative endings: -us,-er,-ius \\\\
\large{\textbf{Examples:}}
\begin{itemize}
  \item servus = slave
  \item dominus = master
  \item filius = son 
  \item puer = boy
  \item ager = field
\end{itemize}
The 2nd declension endings chart \\
\begin{tabular}{llll}
  Case & Singular & Plural & Notes \\ 
  Nominative & -us & -ī & Subject of the sentence \\
  Genitive & -ī & -ōrum & Show possesion ("of the ...") \\ 
  Dative & -ō & -īs & Indirect Object ("to/for the ...") \\ 
  Accusative & -um & -ōs & Direct Object \\ 
  Ablative & -ō & -īs & Object of preposition \\ 
  Vocative & -e / -ī (for ius) & -ī & Direct Address ("O ...") \\
\end{tabular}\\ \\
\large{\textbf{Example:}}\\ \\
\begin{tabular}{lll}
  Case & Singular & Plural \\ 
  Nominative & servus & servī \\ 
  Genitive & servī & servōrum \\ 
  Dative & servō & servīs \\ 
  Accusative & servum & servōs \\ 
  Ablative & servō & servīs \\ 
  Vocative & serve & servī \\
\end{tabular} \hspace{1cm}
\begin{tabular}{lll}
  Case & Singular & Plural \\ 
  Nominative & fīlius & fīliī \\ 
  Genitive & fīliī & fīliōrum \\ 
  Dative & fīliō & fīliīs \\ 
  Accusative & fīlium & fīliōs \\ 
  Ablative & fīliō & fīliīs \\ 
  Vocative & fīlī & fīliī \\
\end{tabular}
\subsection*{Neuter Nouns}
Neuter nouns follow the same pattern in many cases, except for one rule \\ \\
\textbf{Neuter Law:} \\ 
In all neuter nouns, the nominative,accusative and vocative are identical in all numbers\\
\begin{tabular}{lll}
 Case & Singular & Plural \\
 Nominative & -um & -a \\ 
 Genitive & -ī & -ōrum \\ 
 Dative & -ō & -īs \\ 
 Accusative & -um & -a \\ 
 Ablative & -ō & -īs \\ 
 Vocative & -um & -a
\end{tabular}
\section*{2nd Conjugation}
The second conjugation verbs are those that end in -ēre in 
the infinitve like vidēre (to see) , habēre (to have) and 
monēre (to warn) 
\subsection*{Present Tense Conjugation for 2nd Conjugation Verbs}
\begin{tabular}{lll}
  Person & Singular & Plural \\ 
  1st Person & -eō & -ēmus \\ 
  2nd Person & -ēs & -ētis \\ 
  3rd Person & -et & -ent \\ 
\end{tabular}
\subsection*{Example with vidēre:}
\begin{tabular}{lll}
  Person & Singular & Plural \\ 
  1st Person & videō & vidēmus \\ 
  2nd Person & vidēs & vidētis \\ 
  3rd Person & videt & vident \\
\end{tabular}
\subsection*{How to Form the Present Tense for 2nd Conjugation Verbs:}
\begin{enumerate}
  \item Remove the -ēre from the infinitive 
  \item Add appropriate endings based on the person and number
\end{enumerate}
\section*{Word List}
\begin{minipage}[t]{0.48\linewidth} 
  \subsection*{1st Declension (Feminine):}
  \begin{enumerate}
    \item Puella = girl 
    \item Māter = mother
    \item Fīlia = daughter 
    \item Vīlla = house
    \item Terra = land,earth
    \item Aqua = water
    \item Luna = moon
    \item Schola = school 
    \item Via = road,way
  \end{enumerate}
\end{minipage}
\hfill
\begin{minipage}[t]{0.48\linewidth} 
  \subsection*{2nd Declension (Masculine):}
  \begin{enumerate}
    \item Agricola = farmer
    \item Puer = boy 
    \item Fīlius = son 
    \item Dominus = master 
    \item Servus = slave 
    \item Amicus = friend 
    \item Poeta = poet 
    \item Legatus = envoy,ambassador
    \item Liber = book
    \item Cibus = food
    \item Nauta = sailor
  \end{enumerate}
\end{minipage}
\subsection*{Verbs (Infinitive):}
\begin{minipage}[t]{0.48\linewidth}
  \begin{enumerate}
    \item Portāre = to carry 
    \item Laborāre = to work 
    \item Amāre = to love 
    \item Habitāre = to live 
    \item Colere = to worship,to cultivate
    \item Vīdēre = to see 
    \item Monēre = to warn 
    \item Dūcere = to lead 
  \end{enumerate} 
\end{minipage}
\hfill
\begin{minipage}[t]{0.48\linewidth}
  \begin{enumerate} 
    \setcounter{enumi}{9}
    \item Legere = to read 
    \item Tēnere = to hold
  \end{enumerate} 
\end{minipage}
\newpage
