\lecture{04}{2025 April 10 15:42}{Day 4}
\section*{3rd Declension}
3rd declension is one of the most important and versatile
declension in the latin because it's used for all genders 
and the form can be a bit irregular so it's need to get 
very familiar with the patters
\subsection*{3rd Declension Overview}
\begin{itemize}
  \item The nominative singular form is unpredictable 
  \item The genitive singular form always ends in -is 
  \item The stem of the word is found by removing -is from 
  the genitive singular (very tricky)
  \item Declension endings adds to the stem not the nominative
\end{itemize}
\subsection*{How to Find the Stem}
\begin{enumerate}
  \item Take the genitve singular form 
  \item Remove the -is ending 
  \item What's the left is the stem you'll use for the rest
  of the forms
\end{enumerate}
\textbf{Example:}\\

māter,mātris $ \longrightarrow $ stem: mātr- \\
rēx,rēgis $ \longrightarrow $ stem: rēg-
\subsection*{Endings for the Masculine and Feminine Nouns}
\begin{center}
  \begin{tabular}{lll}
    Case & Singular & Plural \\ 
    Nominative & varies & -ēs \\ 
    Genitive & -is & -um/-ium \\ 
    Dative & -ī & -ibus \\ 
    Accusative & -em & -ēs \\ 
    Ablative & -e & -ibus \\ 
  \end{tabular}
\end{center}
\subsection*{Neuter 3rd Declension}
\begin{center}
  \begin{tabular}{lll}
    Case & Singular & Plural \\ 
    Nominatice & varies & -a \\ 
    Genitive & -is & -um / -ium \\ 
    Dative & -ī & -ibus \\ 
    Accusative & same as nom. & same as nom. \\ 
    Ablative & -e & -ibus \\
  \end{tabular}
\end{center}
\subsection*{Special Notes}
\begin{itemize}
  \item Nominative singular is various you have memorize it 
  \item Adjactives of 3rd declension follow similiar patterns 
  \item Abstract nouns often fall into this declension
\end{itemize}
\subsection*{Examples}
{\centering
\begin{tabular}{|l|c|c|}
\hline
\textbf{Case} & \textbf{Singular} & \textbf{Plural} \\
  \hline
  Nominative & rēx & rēgēs \\
  Genitive   & rēgis & rēgum \\
  Dative     & rēgī & rēgibus \\
  Accusative & rēgem & rēgēs \\
  Ablative   & rēge & rēgibus \\
  \hline
\end{tabular}
\hspace{2mm}
\begin{tabular}{|l|c|c|}
  \hline
  \textbf{Case} & \textbf{Singular} & \textbf{Plural} \\
  \hline
  Nominative & corpus & corpora \\
  Genitive   & corporis & corporum \\
  Dative     & corporī & corporibus \\
  Accusative & corpus & corpora \\
  Ablative   & corpore & corporibus \\
  \hline
\end{tabular}
\newline
\vspace*{8mm}
\newline
\begin{tabular}{|l|c|c|}
  \hline
  \textbf{Case} & \textbf{Singular} & \textbf{Plural} \\
  \hline
  Nominative & mare & maria \\
  Genitive   & maris & marium \\
  Dative     & marī & maribus \\
  Accusative & mare & maria \\
  Ablative   & marī & maribus \\
  \hline
\end{tabular}}
\section*{i-Stem Idenfication Rules}
You’ll identify i-stems by these traits:
